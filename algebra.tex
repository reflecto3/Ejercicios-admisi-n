\section{Métodos para encontrar incógnitas}

\begin{questions}
\question
Mirta, Óscar y Gloria son estudiantes
universitarios. Gloria ganó 50 créditos más que
Óscar. Óscar ganó el triple de créditos que Mirta.

Si entre los tres han ganado más de 78 créditos
pero menos de 99, entonces, es posible que\footnote{\cite{SEMA2021}}
\begin{choices}
    \choice Mirta haya ganado 7 créditos.
    \choice Óscar haya ganado 12 créditos.
    \CorrectChoice Óscar haya ganado 15 créditos. %%%%%%%%%%
    \choice Gloria haya ganado 59 créditos.
    \choice Óscar y Mirta juntos hayan ganado 44 créditos
más que Gloria.
\end{choices}

\begin{solution}
    En esta pregunta, nos piden escoger una opción que sea posible. Es decir, el resto serían imposibles. 

    Analicemos y procesemos la información: si designamos por la inicial de su nombre la cantidad de créditos que ganó cada estudiante, tenemos las siguientes relaciones
    \begin{align*}
        &g = o + 50 &&\text{Gloria ganó 50 créditos más que Óscar.} \\
        &o = 3m &&\text{Óscar ganó el triple de créditos que Mirta.} \\
        &78 < m+o+g < 99 &&\text{Entre los trés ganaron entre 78 y 99 créditos.}
    \end{align*}

    Con esto, analicemos las opciones:
    \begin{enumerate}[A.]
        \item Si $m=7$, entonces por las ecuaciones $o = 3\cdot 7 = 21$ y $g = 21+50 = 71$. Entre los tres por tanto ganaron $m+o+g = 7 + 21+ 71 = 99$. Pero nos dicen que ganaron menos de 99 créditos. Por tanto esta no es posible.
        \item Si $o=12$, entonces $12=o = 3m$ implica que $m=4$ y $g = o+50 =12+50 = 62$. Suman $m+o+g = 4 + 12 + 62 = 78$. Pero nos dicen que ganaron más de 78 créditos. Esta tampoco es posible.
        \item Si $o = 15$, $m = o/3 = 15/3 = 5$ y $g = o+50 = 15 + 50 = 65$. De esta forma $m+o+g = 5 + 15 + 65 = 85$ que sí está entre 78 y 99. Por lo tanto, la opción C. es posible, de forma que es la opción correcta.
        \item Si $g = 59$, $o = g-50 = 9$ y $m = o/3 = 3$. Suman $59+9+3 = 71$ que es menor a 78.
        \item Si Óscar y Mirta juntos ganaron 44 créditos más que Gloria, en ecuación esto es $o+m=44+g$. Pero como $g = o+50$, podemos sustituir la $g$ para obtener $\cancel o+m = 44+\cancel o+50$ y así $m = 94$. Como Óscar tiene el triple de créditos de Mirna, claramente se pasan de los 99 que tienen entre los 3.
    \end{enumerate} \hfill $\square$

    {\small\itshape Una forma alternativa de resolver este problema es resolver el sistema para saber en qué rangos podían andar los creditajes de cada estudiante. Como $g = o+50$ y $o = 3m$, sustituyendo $g = 3m + 50$. Así, tenemos que $m+o+g = m + 3m + 3m+50 = 7m+50$. Como $78< m+o+g< 99$,
    \begin{gather*}
        78 < 7m+50 < 99 \\
        78-50 < 7m < 99-50 \\
        28 < 7m < 49 \\
        \frac{28}{7} = 4 < m < \frac{49}{7} = 7
    \end{gather*}
    Así, $m$ debe estar entre 4 y 7, sin incluirlos. Como $o=3m$, entonces $12 = 3\cdot 4 < o < 21 = 3\cdot 7$. Por último, como $g = o+50$, entonces $12+50 = 62 < g < 21+50 = 71$. Con esto, podemos descartar muy rápidamente la A., la B. y la D. La E. también podemos descartarla, pues como $12 < o < 21$ y $4 < m <7$, entonces $16 < m+o < 28$, que claramenete no puede ser 44 créditos más que $g$.}
\end{solution}

\question % GPT
Una profesora tenía 10\,000 gapes y compró 4 cuadernos y 2 lápices para cada una de sus 5 clases. Sabiendo que cada cuaderno cuesta 200 y cada lápiz 100, ¿cuál de las siguientes expresiones representa el dinero restante?
\begin{choices}
    \choice $10000 - 20\cdot200 - 10\cdot100$.
    \choice $10000 - 5(200+100)$.
    \choice $10000 - 4\cdot200 - 2\cdot100$.
    \choice $10000 - 5(4\cdot200 - 2\cdot100)$.
\end{choices}

\question
En un terreno hay 130 árboles, de los cuales 94 son árboles altos y el resto son medianos. Hay 48 árboles jóvenes y la cuarta parte de los árboles medianos son jóvenes.

¿Cuántos de los árboles altos son viejos?\footnote{\cite{practicaUCR1}}
\begin{choices}
    \choice 27
    \choice 36
    \choice 55
    \choice 82
\end{choices}

\question
El peso de 2 platos es igual al peso de 3 botellas y el peso de 3 vasos es igual al de 2 botellas.

¿Cuántas botellas se necesitan para tener el peso de 8 platos y 6 vasos?\footnote{\cite{practicaUCR1}}
\begin{choices}
    \choice 12
    \choice 16
    \choice 21
    \choice 34
\end{choices}

\question
En una fábrica se empacó cierto producto de forma individual. La fábrica utilizó 2 máquinas para realizar este trabajo. La máquina antigua empacó 24 productos cada hora. La máquina nueva empacó 30 productos cada hora. Ayer la máquina antigua comenzó a empacar a las 7:00 a. m. y la máquina nueva comenzó a empacar a las 8:30 a. m.

¿Qué hora era cuando ambas máquinas llevaban la misma cantidad de producto empacado?\footnote{\cite{practicaUCR1}}
\begin{choices}
    \choice 9:30 a. m.
    \choice 11:30 a. m.
    \choice 2:30 p. m.
    \choice 3:30 p. m.
\end{choices}

\question
Cierto año, Rebeca tenía 20 años y sus dos
hermanos 6 y 7 años.

¿Cuál es el menor número de años que debe
transcurrir, a partir de ese año, para que la edad
de Rebeca llegue a ser menor que la suma de las
edades que tendrán sus dos hermanos?\footnote{\cite{SEMA2021}}
\begin{choices}
    \choice 28
    \choice 16
    \choice 9
    \choice 8 %%%%%%%%
    \choice 7
\end{choices}

\question
En la escuela Aprendamos hay 150 estudiantes, de los cuales 95 son de primer ciclo y el resto de segundo ciclo. Si 70 de los estudiantes son varones y la quinta parte de los estudiantes de segundo ciclo son varones, ¿Cuántos estudiantes de primer ciclo son mujeres? \footnote{\cite{SEMA2021}}
\begin{choices}
    \choice 11
    \choice 36
    \choice 59
    \choice 80
\end{choices}

\question
Si hace 5 años la persona P tenía el cuádruplo de la edad de la persona Z y dentro de 5 años tendrá el doble de la edad de Z, ¿cuántos años tiene P? \footnote{\cite{TEC2023}}
\begin{choices}
    \choice 15
    \choice 20
    \choice 25 %%%%%%%%%%%%
    \choice 30
\end{choices}

\question La diferencia entre dos números enteros positivos es 48. Si se divide el mayor entre el menor el cociente es 6 y el residuo es 3.

¿Cuál deducción es correcta?
\begin{choices}
\choice La suma del número mayor y el número menor da un número impar.
\choice La suma de los dígitos del número mayor es impar.
\choice El séxtuplo del cuadrado del número mayor es un número par.
\choice La suma de las cifras del número menor es par.
\choice El producto de los dígitos del número mayor es par.
\end{choices}

\end{questions}