\section{Razonamiento Lógico Verbal}
Nota para el futuro: Esta sección puede separarse en varias secciones distintas, clasificando los problemas. Actualmente no sé cómo clasificarlos, pero considero bueno tener una lista de problemas de lógica verbal con algunas soluciones preliminares.

\begin{questions}

\question En la Antigüedad, la moneda sirvió no solo para comprar y vender productos
sino también para enaltecer la figura de un dios o un soberano o para ser un
símbolo de algunos imperios. Muchas veces el monto que designaba la moneda
era equivalente al valor que tenia el metal con el cual fue hecha. Sin embargo,
con el tiempo la escasez del oro, la plata y el cobre hizo que se fabricaran las
monedas con metales más baratos,
¿Cuál de las siguientes opciones es una idea implícita del texto? \footnote{\cite{24preguntas}}

\begin{choices}
\choice El valor de la moneda era solo económico.
\choice El valor de la moneda era relativo al metal de fabricación.
\choice El valor de la moneda no estaba determinado por la época.
\CorrectChoice El valor de la moneda no siempre estaba sujeto al material utilizado. %%%%%
\end{choices}

\begin{solution}
    En las preguntas de lógica verbal, conviene leer primero la pregunta, y después leer el texto. Aquí por ejemplo necesitamos encontrar una idea implícita en el texto. Podemos ir descartando opciones.

    La opción A. queda descartada de inmediato: ``(...) la moneda sirvió \textbf{no solo} para comprar y vender productos, sino también para enaltecer la figura de un dios (...)''.

    La opción B. es cierta para algunos periodos, pero no en general. De hecho, por la formulación - parafraseando - ``El monto de la moneda era equivalente al valor que tenía el metal de fabricación. \textbf{Sin embargo}, con el tiempo se hicieron de metales más baratos''. El valor de la moneda pudo haber bajado también, pero por el contraste, se nos da a entender que el valor dejó de ser equivalente al que tenía el metal de fabricación. De hecho, este mismo análisis nos permite concluir que la opción D. es la correcta.

    Por completitud, veamos que la C. no es la correcta, pues en el texto se da a entender que las épocas sí influían en el valor de las monedas: ``(...) con el tiempo la escasez del oro, la plata y el cobre hizo que se fabricaran monedas con metales más baratos''. Al menos, no se da a entender que el valor de la moneda fuese constante. \hfill $\square$
\end{solution}

\question Según una leyenda, había dos pueblos cercanos, uno al sur y el otro al norte,
separados solamente por un río, Durante el invierno, los habitantes del norte viajaban al sur y convivían con la gente del otro pueblo, Después de siglos de hacer
lo mismo, las personas del sur utilizaban mas de tres mil palabras del idioma del
norte. Ese vocabulario se refería a muchos objetos, como nombres de alimentos
y de vestimenta, pero las palabras referidas a todos los sentimientos siguieron
siendo las del idioma del sur.
¿Qué se puede concluir del texto anterior? \footnote{\cite{24preguntas}}

\begin{choices}
    \choice El pueblo del norte viajaba al sur para evitar el frio.
    \CorrectChoice El pueblo del sur adoptó palabras cuyo referente era concreto. %%%%%
    \choice El pueblo del norte no tenia palabras para referirse a los sentimientos.
    \choice El pueblo del sur no se relacionaba emocionalmente con el pueblo del norte.
\end{choices}

\begin{solution}
    Es importante leer primero la pregunta: ¿qué se puede \textit{concluir} del texto anterior? No nos piden inferir o conjeturar, hacer hipótesis, sino escoger una idea que con casi total certeza se derive del texto.

    En este sentido, a pesar de que las respuestas A., C., y D. presentan conjeturas o hipótesis viables - unas más que otras - estas conjeturas no son completamente respaldadas por el texto, y así como podrían ser ciertas, podrían no serlo. 

    En la A., es muy probable que se desplazaran para evitar el frío, por ser invierno, pero podría ser que el frío no fuera el problema - que tuvieran suficiente combustible para calentarse - si no la comida. Esto es una hipótesis alternativa. Es decir, la A. no se concluye directamente del texto.

    Similarmente, la C. y la D. dan conjeturas viables para explicar por qué el pueblo del sur no adoptó palabras referidas a los sentimientos del pueblo del norte. Sin embargo, ambas son válidas pero no se pueden concluir con certeza. Podría ser que los del norte sí tuvieran palabras para los sentimientos, pero no se relacionaran emocionalmente con los del sur. O podría ser que sí se relacionaran emocionalmente con los del sur, pero que estos últimos tuvieran un vocabulario sentimental mucho más desarrollado, de forma que fueran los del norte quienes adoptaron palabras de los del sur. Así, no podemos concluirlas con certeza.

    La B., en cambio sintetiza un aspecto que aparece en el texto: ``Ese vocabulario - norteño adoptado por el pueblo del sur - se refería a muchos \textbf{objetos}, como nombres de alimentos y de vestimenta, pero las palabras referidas a todos los sentimientos siguieron siendo las del sur''. Es decir, las palabras adoptadas se referían a cosas concretas: su referente era concreto. \hfill $\square$
\end{solution}

\question Los incas asimilaron las aportaciones de los pueblos integrados al imperio. Para algunos investigadores, lo "incaico" no equivale a lo "peruano". Además, se considera que la cultura inca debe muchos de sus atributos a otras culturas que podrían encontrarse en lo que actualmente no es Perú. No obstante, dichos estudiosos no se han puesto de acuerdo en relación con el tipo de vínculos que existen entre esas culturas y los incas.

¿Qué no se concluye del texto anterior? \footnote{\cite{24preguntas}}

\begin{choices}
\CorrectChoice Los otros pueblos tenían culturas muy distintas a los incas. %%%%
\choice Los incas se caracterizaban por interactuar con los otros pueblos.
\choice Los investigadores consideran que lo cultural y lo geográfico son disímiles.
\choice Los investigadores carecen de información suficiente para llegar a un acuerdo.
\end{choices}

\begin{solution}
    Leamos primero la pregunta, ¿qué \textbf{no} se concluye del texto anterior?

    En estas es muy fácil que por no leer la pregunta nos equivoquemos por escoger alguna de las ideas que sí se concluyen del texto.

    Como la cultura inca debe muchos de sus atributos a otras culturas, se puede concluir que debió haber interacción entre los inclas y otros pueblos. Así, es posible que la B. se concluya. No es correcta.

    La cultura inca debe muchos de sus atributos a otras culturas fuera del actual Perú (parafraseo). Sí se concluye una distinción entre lo cultural y lo geográfico. Así la C. podría concluirse y por tanto no es correcta.

    Los estudiosos no se han puesto de acuerdo en respecto al tipo de vínculos que existen entre esas culturas y los incas (parafraseo). Aunque no se habla de información, podría concluirse la D. pues, si tuvieran toda la información sobre los vínculos, estarían de acuerdo en lo evidente. 

    Así, debe ser la A., y lo es, pues como ``los incas asimilaron las aportaciones de los pueblos integrados al imperio'', asimilaron también parte de su cultura, como se dice unas oraciones después. Por lo tanto, la cultura de estos otros pueblos, al ser asimilada por los incas, no puede ser muy distinta a la que terminaron teniendo los incas. Así, la A. no se concluye, y debe ser correcta. \hfill $\square$
\end{solution}

\question Los marineros expertos no se hacen en las costas, sino en altamar.

¿Qué se puede afirmar del texto anterior? \footnote{\cite{24preguntas}}
\begin{choices}
    \choice Que los marineros siempre están en altamar.
    \choice Que en las costas no hay expertos marineros.
    \CorrectChoice Que las costas no forman a los expertos marineros.
    \choice Que los expertos marineros están cerca de las costas.
\end{choices}

\begin{solution}
    Podemos analizar las opciones una a una.

    A. El texto afirma que los expertos marineros se hacen en altamar. Pero esto no implica que siempre estén allí. Así, esta no es.

    B. Análogo al análisis de la A., los expertos marineros no se hacen en las costas, pero esto no implica que no puedan estar en las costas.

    C. Esta sí es. En este contexto, "hacer marineros", se puede entender por "formar marineros". Es el proceso mediante el cual se llega a ser marinero. En este sentido, lo que dice el texto es: Los marineros expertos no se forman en las costas. Esto es lo que expresa la opción C.

    D. El texto no dice nada acerca de la ubicación de los expertos marineros, sino de dónde se forman aquellos que aspiran a serlo. Así, la D. no puede ser afirmada a partir del texto. \hfill $\square$
\end{solution}

\question El héroe es una figura que se encuentra en todas las culturas y que se caracteriza por ser excepcional. Tiene la capacidad de mantener la constancia en sus buenas acciones y representa los intereses de un colectivo. Por estas razones se le rinde culto y se idealiza tanto su imagen física como simbólica. Muchos lo consideran un individuo inmortal. \footnote{\cite{24preguntas}}

¿Cuál de las siguientes opciones es una idea implícita del texto anterior?

\begin{choices}
    \choice El héroe practica el bien común.
    \choice El héroe ejerce el poder político.
    \choice El héroe representa un modelo imposible.
    \CorrectChoice El héroe genera la identidad para un pueblo.
\end{choices}

\begin{solution}
    Note que se nos pide una idea implícita del texto. La A. por tanto, no puede ser pues está de forma explícita en el texto.

    La B. no es, pues no se habla de política. No se habla de que el héroe tenga un liderazgo político, pues aunque represente los intereses de un colectivo, no se entiende por esto que lo haga de forma oficial dentro de una institución gubernamental.

    La C. podría darse a entender por el hecho de que muchos lo consideren un individuo inmortal. Sin embargo, se presenta como una exaltación suprema, ideal del héroe, y no de forma que el pueblo al que representa lo vea imposible.

    La opción correcta es la D. El texto nos explica que representa los intereses de un pueblo, pero no solo eso, sino que se le rinde culto y se le idealiza, además de que es una figura universal, se presenta en todas las culturas. Por tanto, a través del héroe se genera la identidad del pueblo. \hfill $\square$
\end{solution}

\question En la biología marina, la medusa es un animal con características únicas que la hacen diferente del resto de los seres vivos. Ella no tiene huesos y su cuerpo está compuesto por un gran porcentaje de agua y solo un 5\% de material sólido. La medusa puede sobrevivir porque tiene receptores nerviosos, ella no envejece y tampoco muere porque siempre se renueva.

¿Cuál de las siguientes opciones presenta una idea del texto anterior, pero con otras palabras? \footnote{\cite{24preguntas}}

\begin{choices}
    \CorrectChoice La medusa se transforma constantemente.
    \choice La biología marina es diferente a la biología terrestre.
    \choice Las características de la medusa la convierten en un ser mortal.
    \choice La composición física de la medusa es irrelevante para su sobrevivencia.
\end{choices}

\begin{solution}
    Nos piden una idea del texto anterior con otras palabras. 

    La A. podría ser, porque que la medusa se transforme constantemente concuerda con la idea de que siempre se renueve. Veamos las demás para descartar que no haya otra mejor.

    La B. no sale del texto. El texto solo habla de la medusa, y no habla de biología terrestre del todo, no ofrece una comparación ni un contraste.

    La C. contradice el texto, pues dice que la medusa nunca muere.

    La D. no puede ser, pues en el texto se da importancia a la composición física de la medusa, y aunque no es explícito que esta le ayude a sobrevivir, tampoco se da a entender que sea irrelevante, más bien al contrario.

    Así, como el resto no son y la A. concuerda, podemos marcarla como la respuesta correcta. \hfill $\square$
\end{solution}


\end{questions}

\theendnotes