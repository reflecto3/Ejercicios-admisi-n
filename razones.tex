\section{Razones y proporciones}

\begin{questions}

\question
El peso de 3 manzanas equivale al de 2 peras, y el de 4 naranjas equivale al de 3 peras. ¿Cuántas naranjas se necesitan para igualar el peso de 9 manzanas?
\begin{choices}
  \choice 6.
  \choice 7.
  \CorrectChoice 8. %%
  \choice 9.
\end{choices}

\begin{solution}
    Tenemos que 3 manzanas equivalen en peso a 2 peras. Por lo tanto, 9 manzanas (que son 3 grupos de 3) equivalen a $3 \cdot 2 = 6$ peras (3 grupos de 2 peras). Ahora, bien, como cada grupito de 3 peras equivalen a 4 naranjas, 6 peras equivalen a $2\cdot 4 = 8$ naranjas. Así, en peso 9 manzanas equivalen a 6 peras que a su vez equivalen a 8 naranjas. La respuesta correcta es la C.
    \begin{align*}
        MMM &= PP \\[1ex]
         MMM &\phantom{=}\ \ PP \\
        \implies MMM &= PP \\
         MMM &\phantom{=}\ \ PP \\[1ex]
         PPP &= NNNN \\[1ex]
         MMM &\phantom{=}\ \ PP \quad\ \  NNN\\
        \implies MMM &= PP = NNN\\
         MMM &\phantom{=}\ \ PP \quad\ \ \ NN \\[1ex]
    \end{align*} \hfill $\square$
\end{solution}

\question
Se necesitan 4 litros de pintura para cubrir 28 $m^2$ de pared. ¿Cuántos litros se necesitan para cubrir 49 $m^2$?
\begin{choices}
    \choice 9 litros
    \choice 8 litros
    \CorrectChoice 7 litros
    \choice 6 litros
\end{choices}

\begin{solution}
    4 litros de pintura cubren 28 $m^2$. Entonces, cada litro cubre una cuarta parte, es decir $28/4 = 7$ $m^4$. Así, para cubrir 49 $m^2$, sabiendo que un litro cubre 7 $m^2$, necesitamos $49/7 = 7$ litros. La respuesta correcta es la C.

    Si lo hacemos con regla de 3, note que la proporcionalidad es directa. Entre más metros cuadrados de pared necesitamos más litros de pintura. Por tanto, decimos ``4 litros es a 28 metros cuadrados como X es a 49 metros cuadrados, y lo escribimos''
    \[
    \frac{4 L}{X} = \frac{28m^2}{49m^2} \implies X = \frac{4L \cdot 49\cancel{m^2}}{28\cancel{m^2}} = \frac{\cancel 4\cdot \cancelto{7}{49}}{\cancel{4}\cdot \cancel 7} L = 7 L.
    \] \hfill $\square$
\end{solution}


\question
Cinco grifos llenan un tanque en 12 horas. ¿Cuántas horas tardarían 3 grifos en llenarlo?
\begin{choices}
    \choice 8 horas
    \choice 15 horas
    \choice 18 horas
    \CorrectChoice 20 horas
\end{choices}


\begin{solution}
    Veamos ahora que entre más grufos estén llenando el tanque, menos tiempo dura este en llenarse. Por lo tanto, la proporcionalidad es inversa. Primero analicémoslo y luego veamos cómo hacer regla de tres. 

    Si cinco grifos duran 12 horas en llenar un tanque, uno solo va a durar 5 veces más, es decir, un grifo dura $12 \cdot 5 = 60$ horas en llenar ese tanque. Esto lo podemos expresar como que se necesitan $60$ horas-grifo para llenar un tanque. Luego, utilizando tres grifos requerimos la tercera parte de esto. Así, con tres grifos llenando el mismo tanque duraríamos $60/3 = 20$ horas. La respuesta correcta es la D. 
    
    Note que en todos los casos, el producto de la cantidad de grifos por la cantidad de horas que duran nos da lo mismo: $5 \cdot 12 = 1 \cdot 60 = 3 \cdot 20$. Esto nos quiere decir que la noción de 60 horas-grifo tiene sentido. Por ejemplo, para hacer 60 horas grifo con 30 grifos necesitamos 2 horas.

    Para plantear una regla de 3 en este caso, hacemos eso mismo. Como la proporcionalidad es inversa, en lugar de dividir multiplicamos. ``5 grifos por 12 horas llenan el mismo tanque que 3 grifos por $X$''. Esto quiere decir que 
    \[5 \text{ grifos} \cdot 12 \text{ horas} = 3\text{ grifos} \cdot X \implies X = \frac{5\cdot 12}{3}h = \frac{60}{3} h = 20 h\] \hfill $\square$
\end{solution}

\question
Ocho obreros construyen una pared en 25 días trabajando 6 horas por día. ¿Cuántos días necesitarán 10 obreros para construir la misma pared si trabajan 5 horas al día?
\begin{choices}
    \choice 25 días
    \choice 27 días 
    \CorrectChoice 24 días
    \choice 22 días
\end{choices}

\begin{solution}
    Tenemos 3 variables: obreros, días y horas por día. Entre más obreros, menos días duran (proporcionalidad inversa). Entre más horas por día, menos días duran (proporcionalidad inversa). Pensémoslo como antes: ``8 obreros trabajando 6 horas por día duran 25 días construyendo una pared''. ¿Cuántas horas-obrero se necesitaron? 6 horas al día por 25 días son $6\cdot25 = 150$ horas cada obrero. Como hay 8 obreros, se necesitaron en total $150\cdot 8 = 1200$ horas-obrero en total para construir la pared. Ahora, si tenemos 10 obreros, para completar $1200$ horas-obrero cada uno debe trabajar $1200/10 = 120$ horas. Como trabajan 5 horas al día, para hacer $120$ horas necesitan $120/5 = 24$ dias. Así, la respuesta correcta es la C.

    Haciéndolo directo: ``8 obreros por 6 horas al día por 25 días hacen lo mismo que 10 obreros por 5 horas al día por X''. Así
    \[
    8 \cdot6\cdot 25 = 10 \cdot 5 \cdot X \implies X = \frac{8\cdot \cancelto{3}{6} \cdot\cancelto{\cancel 5}{25}}{\cancel 5\cdot \cancelto{\cancel 2}{10}} = 24 \text{ días}
    \] \hfill $\square$
\end{solution}

\question
Una receta para preparar 12 galletas utiliza 300 gramos de harina. Si se desea preparar 90 galletas con la misma receta, ¿cuántos kilogramos de harina se necesitan?

\begin{choices}
    \choice 1.5
    \CorrectChoice 2.25
    \choice 2.0
    \choice 1.75
\end{choices}

\begin{solution}
En este caso la proporcionalidad es directa. Podemos decir ``300 gramos divididos en 12 galletas equivalen a X divididos en 90 galletas.'' Es decir:
\[\frac{300g}{12} = \frac{X}{90} \implies X = \frac {90\cdot\cancelto{100}{300}}{\cancelto{4}{12}}g = 4500g/2 = 2250g = 2.25 kg\]
Así que la respuesta correcta es la B.

También podemos hacerla por pasos. Si son 300 gramos para 12 galletas, son $300/12 = 100/4 = 25$ gramos por galleta, y así para 90 galletas se necesitan $25\cdot 90 = 25\cdot100 - 25\cdot 10 = 2500-250 = 2250$ gramos, que son $2.25$ kilogramos. \hfill $\square$
\end{solution}

\question
En una obra, 6 personas pueden construir un muro en 10 días, trabajando al mismo ritmo.

¿Cuántos días tardarían 15 personas en construir el mismo muro?

\begin{choices}
    \CorrectChoice 4
    \choice 5
    \choice 6
    \choice 8
\end{choices}

\begin{solution}
    Es proporcionalidad inversa: ``6 personas por 10 días hacen lo mismo que 15 personas por X''. Para construir el muro se necesitan $6\cdot 10 = 60$ días-persona (i.e. una persona duraría 60 días), luego 15 personas duran $60/15 = 4$ días. La respuesta correcta es la A. \hfill $\square$
\end{solution}

\question
Se preparan jugos usando una mezcla de fruta y agua en razón 2:5. Si se tienen 14 litros de fruta, ¿cuántos litros de agua se necesitan para mantener la misma proporción?

\begin{choices}
    \choice 28
    \choice 21
    \CorrectChoice 35
    \choice 24
\end{choices}

\begin{solution}
    Nos dicen que por cada los litros de fruta y agua están en razón 2:5. Es decir, por cada 2 litros de fruta necesitamos 10 de agua. Así, como 14 litros de fruta son 7 veces 2, entonces vamos a necesitar $7\cdot 5 = 35$ litros de agua para mantener la proporción. En otras palabras,
    \[
    \frac{2}{5} = \frac{14}{35}
    \]
    La respuesta correcta es la C. \hfill $\square$
\end{solution}

\question
Tres grifos abiertos simultáneamente pueden llenar un tanque en 8 horas. Si solo se abren dos de esos grifos, ¿cuánto tardarán en llenarlo?

\begin{choices}
    \choice 10
    \choice 11
    \CorrectChoice 12
    \choice 16
\end{choices}


\begin{solution}
Entre más grifos, sale más agua y por tanto el tiempo se reduce. Es decir, tenemos un caso de proporcionalidad inversa. Si 3 grifos llenan el tanque en 8 horas, entonces el número total de grifo-horas es:

\[
3 \times 8 = 24 \text{ grifo-horas}
\]

Es decir, usando un solo grifo duraríamos 24 horas. Si se abren 2 grifos duraríamos

\[
\frac{24}{2} = 12 \text{ horas}
\]

La opción correcta es la C. \hfill $\square$
\end{solution}




\question
Un taller imprime 360 panfletos en 6 horas con 3 impresoras trabajando al mismo ritmo. Si se quieren imprimir 600 panfletos en 5 horas, ¿cuántas impresoras se necesitan?

\begin{choices}
    \choice 4
    \choice 5
    \CorrectChoice 6
    \choice 7
\end{choices}

\begin{solution}
Este es un caso de proporcionalidad mixta: 360 panfletos en 6 horas por 5 impresoras. Así 360 panfletos requieren $6\cdot3 = 18$ impresora-horas (una sola impresora tardaría 18 horas en imprimir los 360 panfletos). Por tanto, una impresora-hora produce $360/18 = 60/3 = 20$ panfletos. Para hacer 600 panfletos necesitamos por tanto $600/20 = 60/2 = 30$ impresora-horas. Si queremos hacerlo en 5 horas, vamos a necesitar $30/5 = 6$ impresoras. La respuesta correcta es la C.

Recapitulando: 360 panfletos en 6 horas con 3 impresoras nos dice que si usamos solo una impresora duraríamos el triple: 360 panfletos en 18 horas con 1 impresora. Si usamos solo 1 hora, entonces el número de panfletos es 20 panfletos en 1 hora con 1 impresora. Luego, para llegar a 600 panfletos, podríamos hacer 600 panfletos en 30 horas con 1 impresora. Si usamos 5 impresoras, el tiempo se divide entre 5: 600 panfletos en 6 horas con 5 impresoras. \hfill $\square$
\end{solution}


\question\footnote{Ejercicio 21 de \cite{SEMA2021}}
En una fábrica, por cada artículo que termine
una persona trabajadora le entregan 2 bonos.
Por cada 3 bonos le dan un almuerzo gratis.
César tuvo derecho a 18 almuerzos gratis en el
año y no le sobraron bonos.

¿Cuál es el número de artículos que César
entregó ese año?

\begin{choices}
    \choice 3
    \choice 12
    \CorrectChoice 27 %%%%%%%%%%
    \choice 54
    \choice 108
\end{choices}

\begin{solution}
    Lo hacemos yendo para atrás. Los 18 almuerzos gratis equivalen a $18\cdot 3 = 30 +24 = 54$ bonos (cada almuerzo gratis requiere 3 bonos). Como le dan 2 bonos por cada artículo terminado, para obtener 54 bonos necesita haber terminado $54/2 = 27$ artículos. La respuesta correcta es la C. \hfill $\square$
\end{solution}


\question
Patricia quiere comprar un desayuno. Ella tiene monedas tipo P y tipo Q para pagar el desayuno. De las monedas tipo P necesitaría 245. Por cada 7 monedas tipo P necesitaría 5 tipo Q. 

¿Cuántas monedas tipo Q necesita Patricia? \footnote{\cite{practicaUCR1}}
\begin{choices}
    \choice 35
    \choice 49
    \choice 168
    \CorrectChoice 175
\end{choices}

\begin{solution}
    Agrupemos las 245 monedas tipo P en grupos de 7: obtenemos $245/7 = 210/7 + 35/7 = 30 + 5 = 35$ grupos. Cada grupo de 7 monedas tipo P podemos canjearlo por un grupo de 5 monedas tipo Q. Haciendo esto obtenemos $35\cdot 5 = 150 + 25 = 175$ monedas tipo Q. Así, Patricia requiere 175 monedas tipo Q. La respueseta correcta es la D. \hfill $\square$
\end{solution}

\question
Considere las siguientes equivalencias:
\begin{enumerate}[I.]
    \item 10 tazas de agua $= 2000$ ml.
    \item 16 cucharadas de agua $=200$ ml.
\end{enumerate}

¿Cuántas tazas se obtienen de 240 cucharadas de agua? \footnote{\cite{TEC2023}}

\begin{choices}
    \choice 15
    \choice 24
    \choice 30
    \choice 48
\end{choices}

\begin{solution}
    Hacemos las conversiones. Como 16 cucharadas de agua son 200 ml, entonces como $240 / 16 = 120/8 = 60/4 = 15$, 240 cucharadas de agua son 15 veces 16 cucharadas de agua, que equivalen a 15 veces 200 ml, que es $15\cdot200 = 30\cdot100 = 3000 ml$. Luego, cada $2000 ml$ son 10 tazas de agua, y $3000/2000 = 1.5$, entonces 3000 mililitros son una y media veces 2000 mililitros, y por tanto equivale a $1.5 \cdot 10 = 15$ tazas de agua. La respuesta es la A.

    Otra forma de pensarlo: como $2000ml = 200ml \cdot 10$, entonces 10 tazas de agua equivalen a $10\cdot 200ml$ que equivale a $10\cdot 16= 160$ cucharadas de agua (pues cada 200 ml son 16 cucharadas de agua). Y ahora sí, usando regla de tres
    \[
    \frac{240\text{ cucharadas}}{160 \text{ cucharadas}} = \frac{X}{10 \text{ tazas}} \implies X = \frac{10\cdot 240}{160} = \frac{240}{16} = \frac{120}{8} = \frac{60}{4} = 15 \text{ tazas.}
    \] \hfill $\square$
\end{solution}

\end{questions}

\printnotes
